\documentclass[leqno]{article}
%\documentclass{article}
\usepackage{breqn}

\fboxsep0pt

\newcommand{\cmd}[1]{%
   \par\noindent%
   \fbox{%
     \parbox{\textwidth}{%
       \begin{dmath}%
         #1%
       \end{dmath}}}}

\begin{document}

% text from email

Dear Morten H\o gholm,

I am using breqn package to typeset equation. It works well. However, I
find there are something wrong when using leqno option in my document.
The following is the description and codes to reproduce the problem.

I just define a new command (\verb|\cmd|) that puts a frame box around a dmath
environment provided by breqn package. I test \verb|\cmd| using the code
attached in the post.

If I don't set leqno option, everything will be OK. However, If I set
leqno option, the width of the second box is not equal to \verb|\textwidth|.
But, the first and the last equation have a normal frame box.

I can't figure out what's the matter. Any suggestions? Thank you very much.

Regards,
Jinsong


\cmd{3\pi}

\cmd{\left(y+x\right)\,\left(y^4-x\,y^3+x^2\,y^2-x^3\,y+x^4\right)\,
  \left(y^6-x\,y^5+x^2\,y^4-x^3\,y^3+x^4\,y^2-x^5\,y+x^6\right)\,
  \left(y^{24}+x\,y^{23}-x^5\,y^{19}-x^6\,y^{18}-x^7\,y^{17}-x^8\,y^{
  16}+x^{10}\,y^{14}+x^{11}\,y^{13}+x^{12}\,y^{12}+x^{13}\,y^{11}+x^{
  14}\,y^{10}-x^{16}\,y^8-x^{17}\,y^7-x^{18}\,y^6-x^{19}\,y^5+x^{23}\,
  y+x^{24}\right)}

\noindent\fbox{%
   \parbox{\textwidth}{%
\cmd{\left(y+x\right)\,\left(y^4-x\,y^3+x^2\,y^2-x^3\,y+x^4\right)\,
  \left(y^6-x\,y^5+x^2\,y^4-x^3\,y^3+x^4\,y^2-x^5\,y+x^6\right)\,
  \left(y^{24}+x\,y^{23}-x^5\,y^{19}-x^6\,y^{18}-x^7\,y^{17}-x^8\,y^{
  16}+x^{10}\,y^{14}+x^{11}\,y^{13}+x^{12}\,y^{12}+x^{13}\,y^{11}+x^{
  14}\,y^{10}-x^{16}\,y^8-x^{17}\,y^7-x^{18}\,y^6-x^{19}\,y^5+x^{23}\,
  y+x^{24}\right)}%
}}

\noindent\rule\textwidth{1mm}


\end{document}
%%% Local Variables: 
%%% mode: latex
%%% TeX-master: t
%%% End: 
