\documentclass{article}
\title{The \pkg{breqn} package --- Alpha Version, 1997/10/30,
1998/10/01, 2001/09/13} 
\author{Michael Downes}
\date{American Mathematical Society}

\pagestyle{headings}
% Cut down the font sizes a bit
\let\Huge\Large \let\huge\Large \let\LARGE\Large
\let\Large\large \let\large\normalsize

\usepackage{verbatim}

\makeatletter
\newwrite\tmp@out
\newcounter{xio}
\newenvironment{xio}{% example input and output
  \par\addvspace\bigskipamount
  \hbox{\itshape 
    \refstepcounter{xio}\kern-\p@ Example \thexio}\@nobreaktrue
  \immediate\openout\tmp@out\jobname.tmp \relax
  \begingroup
  \let\do\@makeother\dospecials\catcode`\^^M\active
  \def\verbatim@processline{%
    \immediate\write\tmp@out{\the\verbatim@line}}%
  \verbatim@start
}{%
  \immediate\closeout\tmp@out
  \@verbatim\frenchspacing\@vobeyspaces
  \@@input \jobname.tmp \relax
  \endgroup
  \par\medskip
  \noindent\ignorespaces
  \@@input \jobname.tmp \relax
  \par\medskip
}
\makeatother

\hfuzz2pc \vbadness9999 \hbadness5000
\def\AmS{{\protect\usefont{OMS}{cmsy}{m}{n}%
  A\kern-.1667em\lower.5ex\hbox{M}\kern-.125emS}}
\def\latex/{{\protect\LaTeX}}
\def\amslatex/{{\protect\AmS-\protect\LaTeX}}
\def\tex/{{\protect\TeX}}
\def\amstex/{{\protect\AmS-\protect\TeX}}

\newcommand{\ntt}{\normalfont\ttfamily}
\chardef\ttbackslash=92
\DeclareRobustCommand{\cs}[1]{{\ntt\ttbackslash#1}}
\let\cn=\cs
\DeclareRobustCommand{\pkg}[1]{{\ntt#1}}
\let\opt=\pkg \let\env=\pkg \let\fn=\pkg

\newcommand\dash{\textemdash}
\newcommand\dbslash[1]{\texttt{\string\\}}
\newcommand\thepkg[1]{the \pkg{breqn} package}

\providecommand\dotsc{\ldots}
\providecommand\eqref[1]{(\ref{#1})}

\providecommand{\qq}[1]{\textquotedblleft#1\textquotedblright}

%\usepackage[cmbase,mathstyleoff]{flexisym}
\usepackage[cmbase]{flexisym}
\usepackage{breqn}

\begin{document}
\maketitle

\section{To do}

\begin{itemize}
\item Handling of QED
\item Space between \verb'\end{dmath}' and following punctuation will
prevent the punctuation from being drawn into the equation.
\item Overriding the equation layout
\item Overriding the placement of the equation number
\item \qq{alignid} option for more widely separated equations where
  shared alignment is desired (requires two passes)
\item Or maybe provide an \qq{alignwidths} option where you give
  lhs/rhs width in terms of ems? And get feedback later on discrepancies
  with the actual measured contents?
\item \cn{intertext} not needed within dgroup! But currently there are
  limitations on floating objects within dgroup.
\item \verb'align={1}' or 2, 3, 4 expressing various levels of demand
  for group-wide alignment. Level 4 means force alignment even if some
  lines then have to run over the right margin! Level 1, the default,
  means first break LHS-RHS equations as if it occurred by itself, then
  move them left or right within the current line width to align them if
  possible. Levels 2 and 3 mean try harder to align but give up if
  overfull lines result.
\item Need an \cs{hshift} command to help with alignment of
  lines broken at a discretionary times sign. Also useful for adjusting
  inside-delimiter breaks.
\end{itemize}

\section{Introduction}

The \pkg{breqn} package for \latex/ provides solutions to a number of
common difficulties in writing displayed equations and getting
high-quality output. For example, it is a well-known inconvenience that
if an equation must be broken into more than one line, \cn{left} \dots\
\cn{right} constructs cannot span lines. The \pkg{breqn} package makes
them work as one would expect whether or not there is an intervening
line break.

The single most ambitious goal of the \pkg{breqn} package, however, is
to support automatic linebreaking of displayed equations. Such
linebreaking cannot be done without substantial changes under the hood
in the way math formulas are processed. For this reason, especially in
the alpha release, users should proceed with care and keep an eye out
for unexpected glitches or side effects.

\section{Principal features}
The principal features of the \pkg{breqn} package are:
\begin{description}

\item[semantically oriented structure] The way in which compound
displayed formulas are subdivided matches the logical structure more
closely than, say, the standard \env{eqnarray} environment. Separate
equations in a group of equations are written as separate environments
instead of being bounded merely by \dbslash/ commands. Among other
things, this clears up a common problem of wrong math symbol spacing at
the beginning of continuation lines. It also makes it possible to
specify different vertical space values for the space between lines of a
long, broken equation and the space between separate equations in a
group of equations.

\item[automatic line breaking] Overlong equations will be broken
automatically to the prevailing column width, and continuation lines
will be indented following standard conventions.

\item[line breaks within delimiters] Line breaks within \cn{left} \dots\
\cn{right} delimiters work in a natural way. Line breaks can be
forbidden below a given depth of delimiter nesting through a package
option.

\item[mixed math and text] Display equations that contain mixed
math and text, or even text only, are handled naturally by means of a
\env{dseries} environment that starts out in text mode instead of math
mode.

\item[ending punctuation] The punctuation at the end of a displayed
equation can be handled in a natural way that makes it easier to promote
or demote formulas from\slash to inline math, and to apply special
effects such as adding space before the punctuation.

\item[flexible numbering] Equation numbering is handled in a natural
way, with all the flexibility of the \pkg{amsmath} package and with no
need for a special \cn{nonumber} command.

\item[special effects] It is easy to apply special effects to individual
displays, e.g., changing the type size or adding a frame.

\item[using available space] Horizontal shrink is made use of
whenever feasible. With most other equation macros it is frozen when it
occurs between \cn{left} \dots\ \cn{right} delimiters, or in any sort of
multiline structure, so that some expressions require two lines that would
otherwise fit on one.

\item[high-quality spacing] The \cn{abovedisplayshortskip} is used when
applicable (other equation macros fail to apply it in equations of more
than one line).

\item[abbreviations] Unlike the \pkg{amsmath} equation environments, the
\pkg{breqn} environments can be called through user-defined abbreviations
such as \cn{beq} \dots\ \cn{eeq}.

\end{description}

\section{Shortcomings of the package}
The principal known deficiencies of the \pkg{breqn} package are:

\subsection{Incompatibilities} As it pushes the envelope
of what is possible within the context of \latex/2e, \thepkg/ will tend
to break other packages when used in combination with them, or to fail
itself, when there are any areas of internal overlap; successful use may
in some cases depend on package loading order.

\subsection{Indention of delimited fragments} When line breaks within
delimiters are involved, the automatic indention of continuation lines
is likely to be unsatisfactory and need manual adjustment. I don't see
any easy way to provide a general solution for this, though I have some
ideas on how to attain partial improvements.

\subsection{Math symbol subversion}
In order for automatic line breaking to work, the operation of all the
math symbols of class 2, 3, 4, and 5 must be altered (relations, binary
operators, opening delimiters, closing delimiters). This is done by an
auxiliary package \pkg{flexisym}. As long as you stick to the advertised
\latex/ interface for defining math symbols (\cn{DeclareMathSymbol}),
things should work OK most of the time. Any more complex math symbol
setup is quite likely to quarrel with the \pkg{flexisym} package.
See Section~\ref{flexisym} for further information.

\subsection{Subscripts and superscripts}

Because of the changes to math symbols of class 2--5, writing certain
combinations such as \verb'^+' or \verb'_\pm' or \verb'^\geq' without
braces would lead to error messages; (The problem described here
already exists in standard \latex/ to a lesser extent, as you may know
if you ever tried \verb'^\neq' or \verb'^\cong'; and indeed there are
no examples in the \latex/ book to indicate any sanction for omitting
braces around a subscript or superscript.)

The \pkg{flexisym} package therefore calls, as of version 0.92, another
package called \pkg{mathstyle} which turns \verb'^' and \verb'_' into
active characters. This is something that I believe is desirable in any
case, in the long run, because having a proper mathstyle variable
eliminates some enormous burdens that affect almost any
nontrivial math macros, as well as many other things where the
connection is not immediately obvious, e.g., the \latex/ facilities for
loading fonts on demand.

Not that this doesn't introduce new and interesting problems of its
own---for example, you don't want to put usepackage statements
after flexisym for any package that refers to, e.g., \verb'^^J' or 
\verb'^^M'
internally (too bad that the \latex/ package loading code does not
include automatic defenses to ensure normal catcodes in the interior of
a package; but it only handles the \verb'@' character).

But I took a random AMS journal article, with normal end-user kind of
\latex/ writing, did some straightforward substitutions to change all
the equations into dmath environments, and ran it with active math
sub/sup: everything worked OK. This suggests to me that it can work in
the real world, without an impossible amount of compatibility work.

\section{Incomplete}
In addition, in the \textbf{alpha release [1997/10/30]} the following
gaps remain to be filled in:
\begin{description}
\item[documentation]
The documentation could use amplification, especially more
illustrations, and I have undoubtedly overlooked more than a few errors.

\item[group alignment] The algorithm for doing alignment
of mathrel symbols across equations in a \env{dgroup} environment 
needs work. Currently the standard and \opt{noalign} alternatives
produce the same output.

\item[single group number] When a \env{dgroup} has a group number and
the individual equations are unnumbered, the handling and placement of
the group number aren't right.

\item[group frame] Framing a group doesn't work, you might be able to
get frames on the individual equations at best.

\item[group brace] The \opt{brace} option for \env{dgroup} is intended
to produce a large brace encompassing the whole group. This hasn't been
implemented yet.

\item[darray environment] The \env{darray} environment is unfinished.

\item[dseries environment] The syntax and usage for the \env{dseries}
environment are in doubt and may change.

\item[failure arrangements] When none of the line-breaking passes for a
\env{dmath} environment succeeds\dash i.e., at least one line is
overfull\dash the final arrangement is usually rather poor. A better
fall-back arrangement in the failure case is needed.

\end{description}

\section{Package options}

Many of the package options for \thepkg/ are the same as options of the
\env{dmath} or \env{dgroup} environments, and some of them require an
argument, which is something that cannot be done through the normal
package option mechanism. Therefore most of the \pkg{breqn} package
options are designed to be set with a \cn{setkeys} command after the
package is loaded. For example, to load the package and set the
maximum delimiter nesting depth for line breaks to~1:
\begin{verbatim}
\usepackage{breqn}
\setkeys{breqn}{breakdepth={1}}
\end{verbatim}

See the discussion of environment options, Section~\ref{envopts}, for
more information.

One package option that may be of interest to \tex/nicians is the
\opt{debug} option; this activates some debugging statements embedded in
the code of the alpha release which trace measuring activities related
to line breaking.

\section{Environments and commands}
\subsection{Environments}
All of the following environments take an optional argument for
applying local effects such as changing the typesize or adding a
frame to an individual equation.

\begin{description}
\item[\env{dmath}] Like \env{equation} but supports line breaking and variant
numbers.

\item[\env{dmath*}] Unnumbered; like \env{displaymath} but supports line
breaking

\item[\env{dseries}] Like \env{equation} but starts out in text mode;
intended for series of mathematical expressions of the form `A, B, and
C'. As a special feature, if you use
\begin{verbatim}
\begin{math} ... \end{math}
\end{verbatim}
for each expression in the series, a suitable amount of inter-expression
space will be automatically added. This is a small step in the direction of
facilitating conversion of display math to inline math, and vice versa: If
you write a display as
\begin{verbatim}
\begin{dseries}
\begin{math}A\end{math},
\begin{math}B\end{math},
and
\begin{math}C\end{math}.
\end{dseries}
\end{verbatim}
then conversion to inline form is simply a matter of removing the
\verb'\begin{dseries}' and \verb'\end{dseries}' lines; the contents of the
display need no alterations.

It would be nice to provide the same feature for \verb'$' notation but
there is no easy way to do that because the \verb'$' function has no
entry point to allow changing what happens before math mode is entered.
Making it work would therefore require turning \verb'$' into an active
character, something that I hesitate to do in a \latex/2e context.

\item[\env{dseries*}] Unnumbered variant of \env{dseries}

\item[\env{dgroup}] Like the \env{align} environment of \pkg{amsmath},
but with each constituent equation wrapped in a \env{dmath},
\env{dmath*}, \env{dseries}, or \env{dseries*} environment instead of being
separated by \dbslash/. The equations are numbered with a group number.
When the constituent environments are the numbered forms (\env{dmath} or
\env{dseries}) they automatically switch to `subequations'-style
numbering, i.e., something like (3a), (3b), (3c), \dots, depending on
the current form of non-grouped equation numbers. See also
\env{dgroup*}.

\item[\env{dgroup*}] Unnumbered variant of \env{dgroup}. If the
constituent environments are the numbered forms, they get normal
individual equation numbers, i.e., something like (3), (4), (5), \dots~.

\item[\env{darray}] Similar to \env{eqnarray} but with an argument like
\env{array} for giving column specs. Automatic line breaking is not
done here.

\item[\env{darray*}] Unnumbered variant of \env{darray}, rather like
\env{array} except in using \cn{displaystyle} for all column
entries.

\item[\env{dsuspend}] Suspend the current display in order to print some
  text, without loss of the alignment. There is also a command form of
  the same thing, \cn{intertext}.
\end{description}

\subsection{Commands}

The commands provided by \thepkg/ are:
\begin{description}
\item[\cn{condition}] This command is used for
a part of a display which functions as a condition on the main
assertion. For example:
\begin{verbatim}
\begin{dmath}
f(x)=\frac{1}{x} \condition{for $x\neq 0$}
\end{dmath}.
\end{verbatim}
\begin{dmath}
f(x)=\frac{1}{x} \condition{for $x\neq 0$}
\end{dmath}.
The \cn{condition} command automatically switches to text mode (so that
interword spaces function the way they should), puts in a comma, and
adds an appropriate amount of space. To facilitate promotion\slash
demotion of formulas, \cn{condition} \qq{does the right thing} if used
outside of display math.

To substitute a different punctuation mark instead of the default comma,
supply it as an optional argument for the \cn{condition} command:
\begin{verbatim}
\condition[;]{...}
\end{verbatim}
(Thus, to get no punctuation: \verb'\condition[]{...}'.)

For conditions that contain no text, you can use the starred form of the
command, which means to stay in math mode:
\begin{verbatim}
\begin{dmath}
f(x)=\frac{1}{x} \condition*{x\neq 0}
\end{dmath}.
\end{verbatim}

If your material contains a lot of conditions like these, you might like
to define shorter abbreviations, e.g.,
\begin{verbatim}
\newcommand{\mc}{\condition*}% math condition
\newcommand{\tc}{\condition}%  text condition
\end{verbatim}
But \thepkg/ refrains from predefining such abbreviations in order that
they may be left to the individual author's taste.

\item[\cn{hiderel}] In a compound equation it is sometimes desired to
use a later relation symbol as the alignment point, rather than the
first one. To do this, mark all the relation symbols up to the desired
one with \cn{hiderel}:
\begin{verbatim}
T(n) \hiderel{\leq} T(2^n) \leq c(3^n - 2^n) ...
\end{verbatim}
\end{description}

\section{Various environment options}\label{envopts}

The following options are recognized for the \env{dmath}, \env{dgroup},
\env{darray}, and \env{dseries} environments; some of the options do not
make sense for all of the environments, but if an option is used where
not applicable it is silently ignored rather than treated as an error.

\begin{verbatim}
\begin{dmath}[style={\small}]
\begin{dmath}[number={BV}]
\begin{dmath}[label={xyz}]
\begin{dmath}[relindent={1em}]
\begin{dmath}[compact]
\begin{dmath}[spread={1pt}]
\begin{dmath}[frame]
\begin{dmath}[frame={1pt},framesep={2pt}]
\begin{dmath}[background={red}]
\begin{dmath}[color={purple}]
\begin{dmath}[breakdepth={0}]
\end{verbatim}

Use the \opt{style} option to change the type size of an individual
equation. This option can also serve as a catch-all option for
altering the equation style in other ways; the contents are simply
executed directly within the context of the equation.

Use the \opt{number} option if you want the number for a particular
equation to fall outside of the usual sequence. If this option is used
the equation counter is not incremented. If for some reason you need to
increment the counter and change the number at the same time, use the
\opt{style} option in addition to the \opt{number} option:
\begin{verbatim}
style={\refstepcounter{equation}}
\end{verbatim}

Use of the normal \cn{label} command instead of the \opt{label} option
works, I think, most of the time (untested).

Use the \opt{relindent} option to specify something other than the
default amount for the indention of relation symbols. The default is
2em.

Use the \opt{compact} option in compound equations to inhibit line
breaks at relation symbols. By default a line break will be taken before
each relation symbol except the first one. With the \opt{compact} option
\latex/ will try to fit as much material as possible on each line, but
breaks at relation symbols will still be preferred over breaks at binary
operator symbols.

Use the \opt{spread} option to increase (or decrease) the amount of
interline space in an equation. See the example given above.

Use the \opt{frame} option to produce a frame around the body of the
equation. The thickness of the frame can optionally be specified by
giving it as an argument of the option. The default thickness is
\cn{fboxrule}.

Use the \opt{framesep} option to change the amount of space separating
the frame from what it encloses. The default space is \cn{fboxsep}.

Use the \opt{background} option to produce a colored background for the
equation body. The \pkg{breqn} package doesn't automatically load the
\pkg{color} package, so this option won't work unless you remember
to load the \pkg{color} package yourself.

Use the \opt{color} option to specify a different color for the contents
of the equation. Like the \opt{background} option, this doesn't work if
you forgot to load the \pkg{color} package.

Use the \opt{breakdepth} option to change the level of delimiter nesting
to which line breaks are allowed. To prohibit line breaks within
delimiters, set this to 0:
\begin{verbatim}
\begin{dmath}[breakdepth={0}]
\end{verbatim}
The default value for breakdepth is 2. Even when breaks are allowed
inside delimiters, they are marked as less desirable than breaks outside
delimiters. Most of the time a break will not be taken within delimiters
until the alternatives have been exhausted.

Options for the \env{dgroup} environment: all of the above, and also
\begin{verbatim}
\begin{dgroup}[noalign]
\begin{dgroup}[brace]
\end{verbatim}

By default the equations in a \env{dgroup} are mutually aligned on their
relation symbols ($=$, $<$, $\geq$, and the like). With the
\opt{noalign} option each equation is placed individually without
reference to the others.

The \opt{brace} option means to place a large brace encompassing the
whole group on the same side as the equation number.

Options for the \env{darray} environment: all of the above (where
sensible), and also
\begin{verbatim}
\begin{darray}[cols={lcr@{\hspace{2em}}lcr}]
\end{verbatim}
The value of the \opt{cols} option for the darray environment should be
a series of column specs as for the \env{array} environment, with the
following differences:
\begin{itemize}
\item For l, c, and r what you get is not text, but math, and
displaystyle math at that. To get text you must use a 'p' column
specifier, or put an \cn{mbox} in each of the individual cells.

\item Vertical rules don't connect across lines.
\end{itemize}

\section{The \pkg{flexisym} package}\label{flexisym}

The \pkg{flexisym} package does some radical changes in the setup for
math symbols to allow their definitions to change dynamically throughout
a document. The \pkg{breqn} package uses this to make symbols of classes
2, 3, 4, 5 run special functions inside an environment such as
\env{dmath} that provide the necessary support for automatic line
breaking.

The method used to effect these changes is to change the definitions of
\cn{DeclareMathSymbol} and \cn{DeclareMathDelimiter}, and then
re-execute the standard set of \latex/ math symbol definitions.
Consequently, additional mathrel and mathbin symbols defined by other
packages will get proper line-breaking behavior if the other package is
loaded after the \pkg{flexisym} package and the symbols are defined
through the standard interface.

\section{Caution! Warning!}
Things to keep in mind when writing documents with \thepkg/:
\begin{itemize}

\item The notation $:=$ must be written with the command \cs{coloneq}.
  Otherwise the $:$ and the $=$ will be treated as two separate relation
  symbols with an \qq{empty RHS} between them, and they will be printed
  on separate lines.

\item Watch out for constructions like \verb'^+' where a single binary
operator or binary relation symbol is subscripted or superscripted. When
the \pkg{breqn} or \pkg{flexisym} package is used, braces are mandatory
in such constructions: \verb'^{+}'. This applies for both display and
in-line math.

\item If you want \latex/ to make intelligent decisions about line
breaks when vert bars are involved, use proper pairing versions of the
vert-bar symbols according to context: \verb'\lvert n\rvert' instead of
\verb'|n|'. With the nondirectional \verb'|' there is no way for \latex/
to reliably deduce which potential breakpoints are inside delimiters
(more highly discouraged) and which are not.

\item If you use the \pkg{german} package or some other package that
turns double quote \verb'"' into a special character, you may encounter
some problems with named math symbols of type mathbin, mathrel,
mathopen, or mathclose in moving arguments. For example, \cn{leq} in a
section title will be written to the \fn{.aux} file as something like
\verb'\mathchar "3214'. This situation probably ought to be improved,
but for now use \cn{protect}.

\item Watch out for the \texttt{[} character at the beginning of a
\env{dmath} or similar environment, if it is supposed to be interpreted
as mathematical content rather than the start of the environment's
optional argument.

This is OK:
\begin{verbatim}
\begin{dmath}
[\lambda,1]...
\end{dmath}
\end{verbatim}
This will not work as expected:
\begin{verbatim}
\begin{dmath}[\lambda,1]...\end{dmath}
\end{verbatim}

\item Watch out for unpaired delimiter symbols (in display math only):
\begin{verbatim}
( ) [ ] \langle \rangle \{ \} \lvert \rvert ...
\end{verbatim}
If an open delimiter is used without a close delimiter, or vice versa,
it is normally harmless but may adversely affect line breaking. This is only
for symbols that have a natural left or right directionality. Unpaired
\cn{vert} and so on are fine.

When a null delimiter is used as the other member of the pair
(\verb'\left.' or \verb'\right.') this warning doesn't apply.

\item If you inadvertently apply \cn{left} or \cn{right} to something
that is not a delimiter, the error messages are likely to be a bit
more confusing than usual. The normal \latex/ response to an error such
as
\begin{verbatim}
\left +
\end{verbatim}
is an immediate message
\begin{verbatim}
! Missing delimiter (. inserted).
\end{verbatim}
When \thepkg/ is in use, \latex/ will fail to realize anything is wrong
until it hits the end of the math formula, or a closing delimiter
without a matching opening delimiter, and then the first message is an
apparently pointless
\begin{verbatim}
! Missing \endgroup inserted.
\end{verbatim}

\end{itemize}

\section{Examples}

\renewcommand\theequation{\thesection.\arabic{equation}}
% Knuth, SNA p74
\begin{xio}
Replace $j$ by $h-j$ and by $k-j$ in these sums to get [cf.~(26)]
\begin{dmath}[label={sna74}]
\frac{1}{6} \left(\sigma(k,h,0) +\frac{3(h-1)}{h}\right)
  +\frac{1}{6} \left(\sigma(h,k,0) +\frac{3(k-1)}{k}\right)
=\frac{1}{6} \left(\frac{h}{k} +\frac{k}{h} +\frac{1}{hk}\right)
  +\frac{1}{2} -\frac{1}{2h} -\frac{1}{2k},
\end{dmath}
which is equivalent to the desired result.
\end{xio}

% Knuth, SNA 4.6.2, p387
\begin{xio}
\newcommand\mx[1]{\begin{math}#1\end{math}}% math expression
%
Now every column which has no circled entry is completely zero;
so when $k=6$ and $k=7$ the algorithm outputs two more vectors,
namely
\begin{dseries}[frame]
\mx{v^{[2]} =(0,5,5,0,9,5,1,0)},
\mx{v^{[3]} =(0,9,11,9,10,12,0,1)}.
\end{dseries}
From the form of the matrix $A$ after $k=5$, it is evident that
these vectors satisfy the equation $vA =(0,\dotsc,0)$.
\end{xio}

\begin{xio}
\begin{dmath*}
T(n) \hiderel{\leq} T(2^{\lceil\lg n\rceil})
  \leq c(3^{\lceil\lg n\rceil}
    -2^{\lceil\lg n\rceil})
  <3c\cdot3^{\lg n}
  =3c\,n^{\lg3}
\end{dmath*}.
\end{xio}

\begin{xio}
The reduced minimal Gr\"obner basis for $I^q_3$ consists of
\begin{dgroup*}
\begin{dmath*}
H_1^3 = x_1 + x_2 + x_3
\end{dmath*},
\begin{dmath*}
H_2^2 = x_1^2 + x_1 x_2 + x_2^2 - q_1 - q_2
\end{dmath*},
\begin{dsuspend}
and
\end{dsuspend}
\begin{dmath*}
H_3^1 = x_1^3 - 2x_1 q_1 - x_2 q_1
\end{dmath*}.
\end{dgroup*}
\end{xio}

\end{document}
